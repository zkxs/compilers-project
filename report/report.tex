\documentclass[paper=letter, fontsize=11pt, oneside, titlepage]{scrartcl}

\usepackage[T1]{fontenc} % 8-bit encoding using fonts that have 256 glyphs
\usepackage{beramono} % A version of Bitstream Vera Mono slightly enhanced for use with TEX
\usepackage{listings} % Required for \lstinputlisting
\usepackage{xcolor} % Required for \definecolor
\usepackage[font=bf,skip=\baselineskip]{caption} % used to set caption style
%\usepackage{sectsty} % Allows customizing section commands
\usepackage{fancyhdr} % Custom headers and footers
\usepackage{nameref} % reference named things by labels. Must be imported before hyperref
\usepackage{hyperref} % allows creation of hyperlinks
\usepackage{enumitem} % for tinkering with spacing between list items


% ------------------------------------------------------------------------------
% Set up page

\addtolength{\textheight}{0.5 in} % make bottom margin a bit smaller

% make side margins a bit smaller
\addtolength{\oddsidemargin}{-0.1in}
\addtolength{\evensidemargin}{-0.1in}
\addtolength{\textwidth}{0.2in}

\setlength\parindent{0pt} % Removes all indentation from paragraphs
\setlength{\parskip}{1em} % Set spacing before paragraphs

\setlist{noitemsep}

% ------------------------------------------------------------------------------
% Set up sections

%\allsectionsfont{\centering} % Make all sections centered


% ------------------------------------------------------------------------------
% Set up code listings

\captionsetup[lstlisting]{font={footnotesize,tt}}

\definecolor{dkgreen}{rgb}{0,0.6,0}
\definecolor{gray}{rgb}{0.5,0.5,0.5}
\definecolor{mauve}{rgb}{0.58,0,0.82}

\lstset{
	basicstyle=\small\ttfamily
}

\lstdefinestyle{myStyle}{
  frame=tb,
  aboveskip=3mm,
  belowskip=3mm,
  showstringspaces=true,
  columns=flexible,
  basicstyle={\footnotesize\ttfamily}, % gives us at least 80 characters per line
  numbers=left,
  numberstyle=\tiny\color{gray},
  keywordstyle=\color{blue},
  commentstyle=\color{dkgreen},
  stringstyle=\color{mauve},
  rulecolor=\color{black},
  frame=single,
  breaklines=true,
  breakatwhitespace=true,
  tabsize=4,
  keepspaces=true,
}

\newcommand{\scalalisting}[2][]{
    \lstinputlisting[style=myStyle, language=scala, caption={\texttt{\detokenize{#2}}},#1]{../src/#2}
}

\newcommand{\datalisting}[2][]{
    \lstinputlisting[style=myStyle, caption={\texttt{\detokenize{#2}}},#1]{../#2}
}

\newcommand{\filelisting}[2][]{
    \lstinputlisting[style=myStyle, title={\texttt{\detokenize{#2}}},#1]{../#2}
}


% ------------------------------------------------------------------------------
% Set up title

\newcommand{\ptitle}{Project 3+4 Report}
\newcommand{\pauthor}{Michael Ripley}
\newcommand{\pschool}{University of Tulsa}
\newcommand{\pteacher}{Dr. Shenoi}
\newcommand{\pclass}{CS-4013-01: Compiler Construction}
\newcommand{\pdate}{\today}

% Create horizontal rule command with 1 argument of height
\newcommand{\horrule}[1]{\rule{\linewidth}{#1}} 

\title{
\vspace{1.8 in} \normalfont \normalsize 
\textsc{\pschool} \\ [25pt] % Your university, school and/or department name(s)
\horrule{0.5pt} \\[0.4cm] % Thin top horizontal rule
\huge \ptitle \\ % The assignment title
\horrule{0.5pt} \\[0.5cm] % Thick bottom horizontal rule
}
\author{\pauthor} % Your name
\date{
    \normalsize
    \pteacher \endgraf
    \pclass \endgraf
    \pdate
} % Today's date or a custom date


% ------------------------------------------------------------------------------
% Set up headers and footers


\pagestyle{fancyplain} % Makes all pages in the document conform to the custom headers and footers
\fancyhead{} % No page header - if you want one, create it in the same way as the footers below
\fancyfoot[L]{} % Empty left footer
\fancyfoot[C]{} % Empty center footer
\fancyfoot[R]{\thepage} % Page numbering for right footer
\renewcommand{\headrulewidth}{0pt} % Remove header underlines
\renewcommand{\footrulewidth}{0pt} % Remove footer underlines
\setlength{\headheight}{13.6pt} % Customize the height of the header


% ------------------------------------------------------------------------------
% Set up hyperref (pdf metadata and hyperlinks)

\hypersetup{
    colorlinks          =    {true},        % hyperlinks will be colored
    urlcolor            =    {Blue},        % url hyperlinks will be blue
    urlbordercolor      =    {0 0 1},       % url borders will be blue
    linkcolor           =    {Black},       % link color is black
    linkbordercolor     =    {0 0 0},       % link border is black
    pdftitle            =    {\ptitle}, 
    pdfauthor           =    {\pauthor},
    pdfcreator          =    {\pauthor},
    pdfproducer         =    {pdfLaTeX},
    pdfdisplaydoctitle  =    {true},
    pdfkeywords         =    {\pauthor} {Assignment} {Computer Science} {Compiler},
    bookmarks           =    {true},
    bookmarksopen       =    {false},
    pdfpagemode         =    {UseNone}
}


\begin{document}

\pdfbookmark[1]{Title Page}{title} % Add PDF bookmark to top of document

\maketitle

\section{Introduction}\label{intro}

Project 3 involves adding a semantic analyzer on top of the existing parser from the previous projects.  The semantic analyzer handles typechecking and declaration processing.  The semantic analyzer must also support some level of error recovery.  

Project 4 involves computing memory addresses for variables in each scope.

\section{Methodology}\label{meth}

To begin project 3 and 4, our massaged grammar from project 2 had to be annotated to include types and declarations.  I performed this annotation twice: once to get the hang of it, and again to fix all the silly things I'd done the first time through.  Type checking was relatively easy, with the notable exception being the slightly more difficult table annotations for ADDOP, MULOP, and RELOP.  With the annotation done, the only thing left is implementing it into the parser. 

\section{Implementation}\label{impl}

To begin, I set up a system for type comparison and declaration processing.  With this framework laid down, I began transcribing the annotations from my grammar into the recursive descent parser.  This was complicated by my error recovery technique, where any production that would typically always return a synthesized attribute could optionally return nothing.  This allowed the parser to continue if a semantic error was encountered, and prevented printing a cascade of errors due to a single problem.  However, it required me to check at each production if I was actually receiving a synthesized attribute from child productions.  Because of this, it took my longer than I expected to completely transcribe my annotations.  

As the lexer already has different token types for INTEGER, REAL, and LONGREAL, obtaining the type of terminals proved relatively simple.  INTEGER required a bit more work, as there are a couple points where the value of IntegerToken is required for use in type checking of arrays.  

Declaration processing was simple once I set up my data structure to contain subprograms and variables.  I ran into a couple of issues at edge cases, such as the allowing of recursive calls, but nothing was too difficult to fix.  

Error recovery uses the same sync sets as project 2 (the follows of each production and EOF), in addition to the previously described optional synthesized attributs.

Finally for project 4, memory addresses had to be computed for variables.  As I had designed my type system with this in mind, it was an extremely simple addition.  It did, however, leave the question of what to do with non-addressed variables (notably procedure parameters).  Addressed variables are printed in the form \lstinline{memloc(0, 72)}, where the first number is the scope index and the second number is the memory offset for that scope.  Non-addressed variables are printed in the form \lstinline{tokidx(8)}, where the number is a simple index starting from zero.  A new .locations output file was added listing the locations of addressed variables.

\section{Discussion and Conclusions}\label{conclusions}

It's pretty cool to have all this working.

\section{References}\label{ref}

\begin{itemize}
    \item Handouts and lecture notes from class
    \item API documentation for both Java and Scala
    \item Discussions between myself, Steven Buchele, and James Rogers
\end{itemize}

\section{Appendix I: Sample Inputs and Outputs}\label{sample}

Two source programs were ran through the parser. The first, ``test1\_correct.pas'', has no errors.  The second, ``test2\_error.pas'', demonstrates that lexical and syntax errors are reported on and recovered from.  

\filelisting[language=pascal]{test1_correct.pas}
\filelisting{test1_correct.pas.listing}
\filelisting{test1_correct.pas.tokens}
\filelisting{test1_correct.pas.locations}
\filelisting[language=pascal]{test2_errors.pas}
\filelisting{test2_errors.pas.listing}
\filelisting{test2_errors.pas.tokens}
\filelisting{test2_errors.pas.locations}

\section{Appendix II: Program Listings}\label{code}

\datalisting{resources/operators.dat}
\datalisting{resources/punctuation.dat}
\datalisting{resources/reservedwords.dat}

\scalalisting{net/michaelripley/pascalcompiler/Compiler.scala}
\scalalisting{net/michaelripley/pascalcompiler/identifiers/Identifier.scala}
\scalalisting{net/michaelripley/pascalcompiler/identifiers/IdentifierError.scala}
\scalalisting{net/michaelripley/pascalcompiler/identifiers/IdentifierManager.scala}
\scalalisting{net/michaelripley/pascalcompiler/identifiers/SubProgram.scala}
\scalalisting{net/michaelripley/pascalcompiler/identifiers/Type.scala}
\scalalisting{net/michaelripley/pascalcompiler/identifiers/TypedIdentifier.scala}
\scalalisting{net/michaelripley/pascalcompiler/lexer/Lexeme.scala}
\scalalisting{net/michaelripley/pascalcompiler/lexer/Lexer.scala}
\scalalisting{net/michaelripley/pascalcompiler/lexer/LineFragment.scala}
\scalalisting{net/michaelripley/pascalcompiler/lexer/LineLocation.scala}
\scalalisting{net/michaelripley/pascalcompiler/lexer/ReservedStrings.scala}
\scalalisting{net/michaelripley/pascalcompiler/parser/Parser.scala}
\scalalisting{net/michaelripley/pascalcompiler/tokenizers/CatchAllTokenizer.scala}
\scalalisting{net/michaelripley/pascalcompiler/tokenizers/CompoundTokenizer.scala}
\scalalisting{net/michaelripley/pascalcompiler/tokenizers/SimpleTokenizer.scala}
\scalalisting{net/michaelripley/pascalcompiler/tokenizers/StringTokenizer.scala}
\scalalisting{net/michaelripley/pascalcompiler/tokenizers/Tokenizer.scala}
\scalalisting{net/michaelripley/pascalcompiler/tokenizers/WordTokenizer.scala}
\scalalisting{net/michaelripley/pascalcompiler/tokens/AttributeToken.scala}
\scalalisting{net/michaelripley/pascalcompiler/tokens/ErrorToken.scala}
\scalalisting{net/michaelripley/pascalcompiler/tokens/IdentifierToken.scala}
\scalalisting{net/michaelripley/pascalcompiler/tokens/IntegerToken.scala}
\scalalisting{net/michaelripley/pascalcompiler/tokens/PartialAttributeToken.scala}
\scalalisting{net/michaelripley/pascalcompiler/tokens/PartialErrorToken.scala}
\scalalisting{net/michaelripley/pascalcompiler/tokens/Token.scala}
\scalalisting{net/michaelripley/Util.scala}

\end{document}
